\documentclass{shortart}

\usepackage{amsmath, amssymb, amsthm}
\usepackage{tikz-cd}
\usepackage[T1]{fontenc}

\title{Construction of synthetic spectra}
\author{Dexter Chua}

\newtheorem{thm}{Theorem}[section]
\newtheorem{lemma}[thm]{Lemma}

\theoremstyle{definition}
\newtheorem{defi}[thm]{Definition}
\newtheorem{remark}[thm]{Remark}
\newtheorem{eg}[thm]{Example}

\newcommand\C{{\mathcal{C}}}
\newcommand\D{{\mathcal{D}}}
\newcommand\Z{{\mathbb{Z}}}
\newcommand\F{{\mathbb{F}}}
\renewcommand\S{{\mathbb{S}}}
\newcommand\Spc{{\mathrm{Spc}}}
\newcommand\Sp{{\mathrm{Sp}}}
\newcommand\op{{\mathrm{op}}}

\DeclareMathOperator\Fun{Fun}
\DeclareMathOperator\Hom{Hom}
\DeclareMathOperator\Ext{Ext}
\DeclareMathOperator\Syn{Syn}
\DeclareMathOperator\Ind{Ind}
\DeclareMathOperator\Mod{Mod}

\DeclareMathOperator*\colim{colim}

\begin{document}
The goal of these notes is to provide motivation for the construction of synthetic spectra. They were originally constructed in \cite{synthetic}, but the story presented here is largely based on \cite{algebraicity}. This is aimed towards readers who are familiar with the formal properties of synthetic spectra and want to understand the construction of the category.

The main idea is that the category of synthetic spectra is the derived category of spectra with respect to $E$-homology. To understand this perspective, we begin with the classical notion of the derived category of an abelian category.

Let $\mathcal{A}$ be an abelian category. One motivation for the derived category is that quotienting often loses information, and we want to somehow remember that information. For example, the cokernel of the map $0\colon \Z \to \Z$ is just $\Z$ itself, and the source has been completely forgotten. In the derived category, we want to ``remember'' this, and the cofiber of this map is the chain complex
\[
  \begin{tikzcd}
    \Z \ar[d, "0"] \\
    \Z.
  \end{tikzcd}
\]
We can think of this as freely adding cokernels, so that in the derived category, the cokernel does not lose information.

However, freely adding \emph{all} cokernels is not quite what we want. For example, the cokernel of $2 \colon \Z \to \Z$ should still be $\Z/2$, because this quotient does not lose any information. That is, we want short exact sequences to remain exact (fiber sequences) in the derived category.\footnote{Projectives and/or injectives enter quite late in this story. In a category with enough projectives, everything can be resolved by projectives, so we can restrict our attention to projectives only. Since short exact sequences of projectives always split, we can omit the second step, and end up with the familiar presentation of the derived category.}

Thus, the construction of the derived category breaks into two steps --- freely add some colimits, and then force certain sequences to remain exact. The construction of synthetic spectra will follow similar footsteps, but with various modifications so that the resulting category has good categorical properties.

\section{Freely adjoining colimits}
We start with the problem of freely adjoining colimits. The simplest case is to freely adjoin \emph{all} small colimits, which gives

\begin{thm}{{\cite[Theorem 5.1.5.6]{htt}}}
  Let $\C$ be a small $\infty$-category. Then the Yoneda embedding $y\colon \C \to P(\C) = \Fun(\C^\op, \Spc)$ is the free co-completion of $\C$. That is, for any co-complete category $\D$, pre-composition with the Yoneda embedding gives an equivalence
  \[
    \Fun^L(P(\C), \D) \to \Fun(\C, \D),
  \]
  where $\Fun^L(P(\C), \D)$ is the category of co-continuous functors $P(\C)\to \D$.
\end{thm}

If we only want to add \emph{some} colimits, we restrict to a subcategory of $P(C)$.

\begin{thm}{{\cite[Proposition 5.3.6.2]{htt}}}
  Let $\C$ be a small $\infty$-category and $\mathcal{K}$ a collection of simplicial sets. Let $P^{\mathcal{K}}(\C)$ be the full subcategory of $P(\C)$ generated by representables under $\mathcal{K}$-indexed colimits. Then for any category $\D$ with $\mathcal{K}$-indexed colimits, pre-composition with the Yoneda embedding gives an equivalence
  \[
    \Fun_{\mathcal{K}}(P^{\mathcal{K}}(\C), \D) \to \Fun(\C, \D),
  \]
  where $\Fun_{\mathcal{K}}(P^{\mathcal{K}}(\C), \D)$ is the category of $\mathcal{K}$-indexed colimit-preserving functors $P^{\mathcal{K}}(\C) \to \D$.
\end{thm}

While the category $P^{\mathcal{K}}(\C)$ exists, it is pretty difficult to reason about in general. Given a presheaf, there is no clear criterion one can use to check whether it is in $P^{\mathcal{K}}(\C)$. Consequently, it is also difficult to prove categorical properties of $P^{\mathcal{K}}(\C)$, e.g.\ if it is presentable.

Thankfully, in certain cases of interest, we can describe $P^{\mathcal{K}}(\C)$ as the category of presheaves that preserve certain limits. Combining \cite[Lemmas 5.5.4.16-18]{htt}, we learn that such categories are accessible localizations of $P(\C)$, and in particular presentable. There are two main such examples:

\begin{thm}[{\cite[Corollary 5.3.5.4]{htt}}]
  Let $\C$ be a small $\infty$-category with finite colimits and $\mathcal{K}$ be the collection of filtered simplicial sets. Then $P^{\mathcal{K}}(\C)$ is the full subcategory of $P(\C)$ consisting of finite limit-preserving presheaves. That is, it sends finite colimits in $\C$ to finite limits in $\Spc$. Moreover, the Yoneda embedding $\C \hookrightarrow P^{\mathcal{K}}(\C)$ preserves all finite colimits.
\end{thm}
In this case, we refer to $P^{\mathcal{K}}(\C)$ as $\Ind(\C)$.

\begin{thm}[{\cite[Lemma 5.5.8.14, Proposition 5.5.8.10]{htt}}]
  Let $\C$ be a small $\infty$-category with finite coproducts and $\mathcal{K}$ be the collection of filtered simplicial sets and $\Delta^\op$. Then $P^{\mathcal{K}}(\C)$ is the full subcategory of $P(\C)$ consisting of (finite) product-preserving presheaves. Moreover, the Yoneda embedding $\C \hookrightarrow P^{\mathcal{K}}(\C)$ preserves all finite coproducts.
\end{thm}
In this case, we refer to $P^{\mathcal{K}}(\C)$ as $P_\Sigma(\C)$.

The is that finite colimits are ``complementary'' to filtered colimits, while coproducts are ``complementary'' to (filtered colimits + geometric realization). Specifically, the first result follows from the facts that
\begin{enumerate}
  \item filtered colimits commute with finite limits in $\Spc$; and
  \item every colimit is a filtered colimit of finite colimits.
\end{enumerate}

\begin{remark}
  Recall that our original motivation was to freely add cokernels to an abelian category. In a non-abelian setting, we would want to add coequalizers, or rather their derived analogues --- geometric realizations. This approach is taken by \cite{algebraicity}, but results in a less pretty category. Our approach here is slightly different, and is based on the ideas of \cite[Section 6.4]{algebraicity}.

  The main observation is that in most cases, our category $\C$ is generated freely by its compact objects $\C^\omega$. That is, we have $\C = \Ind(\C^\omega)$. Instead of freely adding filtered colimits to $\C^\omega$, then freely adding geometric realizations, a better strategy is to start with $C^\omega$ and freely add filtered colimits \emph{and} geometric realizations in one go. The restricted Yoneda functor $\C \to P_\Sigma(\C^\omega)$ is easily seen to be fully faithful and preserve filtered colimits. Since $\C^\omega$ is usually essentially small, this also lets us avoid size issues.
\end{remark}

\begin{proof}
  We prove the case of $\Ind(\C)$. The proof for $P_\Sigma(\C)$ is similar. To disambiguate, let $\Ind(\C) \subseteq P(\C)$ be the category of finite limit-preserving sheaves.

  We first show that $y\colon \C \to \Ind(\C)$ preserves finite colimits. This follows from a straightforward calculation
  \[
    \begin{aligned}
      \Hom\left(\colim y(P_\alpha), X\right) &= \lim \Hom(y(P_\alpha), X)\\
                                             &= \lim X(P_\alpha)\\
                                             &= X\left(\colim P_\alpha\right) \\
                                             &= \Hom\left(\colim y(P_\alpha), X\right).
    \end{aligned}
  \]

  Since filtered colimits of spaces commute with finite limits, we know that $\Ind(\C)$ is closed under filtered colimits. Since representables preserve finite limits, we know that $P^{\mathcal{K}}(\C) \subseteq \Ind(\C)$.

  To show the other inclusion, let $X \in P(\C)$. Then we can write
  \[
    X = \colim_{j \in \mathcal{J}} X_j,
  \]
  where $\mathcal{J}$ is filtered and each $X_j$ is a finite colimit of representables. Now suppose that $X \in \Ind(\C)$. Our goal is to write $X$ as a filtered colimit of representables.

  Let $\iota \colon \Ind(\C) \hookrightarrow P(\C)$ be the inclusion, and $L \colon P(\C) \to \Ind(\C)$ its left adjoint. Then they both preserve filtered colimits, and
  \[
    X = \iota LX = \colim_{j \in \mathcal{J}} \iota L X_j.
  \]
  So it suffices to show that $\iota L X_j$ is representable. Let $X_j = \colim y(P_\alpha)$. Then we have
  \[
    LX_j = L \colim y(P_\alpha) = \colim y(P_\alpha),
  \]
  where the second colimit is taken inside $\Ind(\C)$. But $y \colon \C \to \Ind(\C)$ preserves finite colimits, so the right-hand side is simply given by $y\left(\colim P_\alpha\right)$.
\end{proof}

\section{The category of synthetic spectra}
Following our previous outline, to construct the category of synthetic spectra, we start with $P_\Sigma(\Sp^\omega)$, and then for every fiber sequence $A \to B \to C$ such that the second map is $E_*$ surjective, we force the image in $P_\Sigma(\Sp^\omega)$ to be a fiber sequence.

In practice, there are some modifications we want to perform. Firstly, we want the category of synthetic spectra to be stable. This can be fixed by simply stabilizing $P_\Sigma(\Sp^\omega)$, and we have the following result:

\begin{thm}
  Let $\C$ be a small $\infty$-category with finite coproducts. Let $P_\Sigma^\Sp(\C)$ be the full subcategory of $\Fun(\C^\op, \Sp)$ consisting of product-preserving functors. Then $P_\Sigma^\Sp(\C)$ is the stabilization of $P_\Sigma(\C)$.
\end{thm}

To impose our second condition, we have to confront ourselves with the unfortunate fact that $E_*$ surjections are not closed under tensor products. For example, the map $S \to S/2$ is $(H\Z)_*$-surjective, but it is not after tensoring with $S/2$. This will cause the resulting category to not have a symmetric monoidal structure. To avoid this, we make the following definition.

\begin{defi}
  Let $E$ be a homotopy ring spectrum. We let $\Sp_E^{fp} \subseteq \Sp^\omega$ be the full subcategory of spectra $P$ such that $E_*P$ is a projective $E_*$-module.
\end{defi}

If $P \in \Sp_E^{fp}$, then for any other $Y$, we have $E_*(P \otimes Y) = E_* P \otimes_{E_*} E_* Y$. So we learn that
\begin{enumerate}
  \item $\Sp_E^{fp}$ is closed under tensor products; and
  \item $E_*$-surjections are closed under tensor products in $\Sp_E^{fp}$.
\end{enumerate}

Replacing $\Sp^\omega$ with $\Sp_E^{fp}$ should not be seen as a big change. In the case $E = H\F_p$, the two categories are equal, so there is literally no difference. In general, $\Sp_E^{fp}$ importantly contains the spheres, from which we can build all other finite spectra.

Thus, our starting category is $P_\Sigma^\Sp(\Sp_E^{fp})$. We impose our epimorphism condition as follows:

\begin{defi}
  We define $\Syn_E$ to be the full subcategory of $P_\Sigma^\Sp(\Sp_E^{fp})$ consisting of functors $X \colon (\Sp_E^{fp})^\op \to \Sp$ such that for any cofiber sequence
  \[
    A \to B \to C
  \]
  of spectra living in $\Sp_E^{fp}$ that induces a short exact sequence on $E_*$-homology, the induced sequence
  \[
    X(C) \to X(B) \to X(A)
  \]
  is a fiber sequence of spectra.
\end{defi}

\begin{remark}
  Since $\Sp$ is stable, this is equivalent to requiring that $X(C) \to X(B) \to X(A)$ is a cofiber sequence. However, if we work with the non-stabilized version $P_\Sigma(\Sp_E^{fp})$, being a fiber sequence is the correct condition.
\end{remark}
\begin{remark}
  We can turn $\Sp_E^{fp}$ into a site by declaring coverings to be generated by $E_*$ surjections. Then $\Syn_E$ is exactly the presheaves that are sheaves under this topology. In particular, $\Syn_E$ is an accessible left exact localization of $P_\Sigma^\Sp(\Sp_E^{fp})$.
\end{remark}

We can write down some examples of synthetic spectra. Define the spectral Yoneda embedding $Y \colon \Sp \to \Syn_E$ by
\[
  Y(X)(P) = F(P, X).
\]
Then for any $X \in \Sp$, we see that $Y(X)$ is in fact a sheaf (i.e.\ in $\Syn_E$). We should think of this as a \emph{bad} thing. Since we didn't use anything about $E$ to conclude that $Y(X)$ is a sheaf, it cannot possibly contain much information about the $E$-based Adams spectral sequence.

This turns out to be the less useful version of the Yoneda embedding. Instead, we define $y \colon \Sp \to P_\Sigma^\Sp(\Sp_E^{fp})$ by
\[
  y(X)(P) = \tau_{\geq 0} F(P, X).
\]
We should think of this as $\Sigma^\infty$ of the usual Yoneda embedding, characterized by the fact that it takes values in connective spectra and $\Omega^\infty y(X)(P) = \Sp(P, X)$. In fact, Yoneda's lemma implies that if $P \in \Sp_E^{fp}$, then
\[
  P_\Sigma^\Sp(\Sp_E^{fp})(y(P), Z) = \Omega^\infty Z(P).
\]

Crucially, $y(X)$ is not always a sheaf! Given a cofiber sequence
\[
  A \to B \to C
\]
in $\Sp_E^{fp}$, the induced sequence
\[
  \tau_{\geq 0}  F(C, X) \to \tau_{\geq 0} F(B, X) \to \tau_{\geq 0} F(A, X)
\]
is a cofiber sequence if and only if the map $[B, X] \to [A, X]$ is surjective.

There is one case where this is in fact a sheaf. If $X$ is $E$-injective, then the map $[B, X] \to [A, X]$ is given by
\[
  \Hom_{E_*E} (E_* B, E_* X) \to \Hom_{E_* E} (E_*A, E_* X).
\]
Since $E_* X$ is an injective $E_*E$-comodule and $E_* A \to E_*B$ is an injective map, it follows that this map is in fact surjective. So
\begin{thm}
  If $X$ is $E$-injective, then $y(X)$ is a sheaf.
\end{thm}

In general, we define
\begin{defi}
  For $X \in \Sp$, we define $\nu X$ to be the sheafification of $y(X)$.
\end{defi}

Since sheafification is left adjoint to the inclusion, for $P \in \Sp_E^{fp}$ and $X \in \Syn_E$, we have
\[
  \Syn_E(\nu P, X) = \Omega^\infty X(P).
\]

\begin{lemma}[{\cite[Lemma 4.23]{synthetic}}]
  If $A \to B \to C$ is a cofiber sequence of spectra that induces a short exact sequence on $E_*$-homology, then
  \[
    \nu A \to \nu B \to \nu C
  \]
  is a cofiber sequence.
\end{lemma}
If these spectra are in $\Sp_E^{fp}$, then this follows from the definition of a sheaf plus the identification $\Syn_E(\nu P, X) = \Omega^\infty X(P)$. The general case requires more work, but is still true nonetheless.

Combining these two results, what we learn is that to compute $\nu X$ for any $X$, we resolve $X$ by $E$-injectives as in the Adams resolution, and then apply $\nu$ to this resolution. This remains a resolution in $\Syn_E$ (barring convergence issues), and $\nu$ of $E$-injectives are simply given by the connective Yoneda embedding. This is what makes $\nu$ much more interesting than $Y$.

Since the tensor product preserves sums and $E_*$-epimorphisms, we find that
\begin{thm}
  $\Syn_E$ is a symmetric monoidal category, and $\nu\colon \Sp_E^{fp} \to \Syn_E$ is symmetric monoidal. In particular, $\S \equiv \nu S$ is the unit.
\end{thm}

\section{The map \texorpdfstring{$\tau$}{tau}}
We define a bigrading on $\Syn_E$ by setting
\[
  (\Sigma^{a, b} X)(P) = \Sigma^{-b} X(\Sigma^{-a - b} P).
\]
The precise combinations on the right are chosen for the purpose of agreeing with the Adams grading in the Adams spectral sequence. Under this grading convention, categorical suspension is $\Sigma^{1, -1}$, while $\nu \Sigma = \Sigma^{1, 0} \nu$. We write $\S^{a, b} = \Sigma^{a, b} \S$.

The main theorem is
\begin{thm}
  There is a map $\tau \colon \S^{0, -1} \to \S$ with the property that
  \begin{enumerate}
    \item There is a fully faithful embedding $\Mod_{C\tau} \to \operatorname{Comod}_{E_*E}$ that sends $C\tau \otimes \nu X$ to $E_* X$.
    \item There is an equivalence of categories $\Mod_{\tau^{-1} \S} \cong \tau^{-1} \S$ that sends $\tau^{-1} \nu X$ to $X$.
  \end{enumerate}
\end{thm}
The $\tau$-Bockstein spectral sequence for $\nu X$ then has $E_2$-page given
\[
  E^{s, t}_2 = \Ext_{E_*E}^{s, t}(E_*, E_*X)
\]
and converges to $\pi_{t - s} X$. Unsurprisingly, this is the Adams spectral sequence for $X$.

We begin by constructing $\tau$, which is in fact a natural transformation
\[
  \tau \colon \Sigma^{0, -1} X \to X.
\]
Fix $X \in P_\Sigma^\Sp(\Sp_E^{fp})$, and let $P \in \Sp_E^{fp}$. In $\Sp_E^{fp}$, we have a pushout diagram
\[
  \begin{tikzcd}
    P \ar[r] \ar[d] & * \ar[d] \\
    * \ar[r] & \Sigma P.
  \end{tikzcd}
\]
Applying $X$ to this diagram, we get
\[
  \begin{tikzcd}
    X(P) & * \ar[l] \\
    * \ar[u] & X(\Sigma P). \ar[u] \ar[l]
  \end{tikzcd}
\]
There is nothing that requires this to be a pushout diagram, but we get a comparison map
\[
  \Sigma X(\Sigma P) \to X(P).
\]
This is exactly the map $\tau$ we seek.

\begin{remark}
  One can show that for any $X \in \Syn_E$, the map $\tau \colon \Sigma^{0, -1} X \to X$ is the tensor product of $X$ with $\tau \colon \S^{0, -1} \to \S$.
\end{remark}

\begin{eg}
  Recall that $y(X)(P) = \tau_{\geq 0} F(P, X)$. Then
  \[
    (\Sigma^{0, -1} y(X))(P) = \Sigma \tau_{\geq 0} F(\Sigma P, X) = \tau_{\geq 1} F(P, X),
  \]
  and $\tau$ is the natural covering map. So $y(X) / \tau = \pi_0 F(P, X)$ while $\tau^{-1} y(X) = Y(X)$.
\end{eg}

We now quickly look at modules over $\tau^{-1} \S$ and $C\tau$.
\begin{defi}
  A synthetic spectrum $X \in \Syn_E$ is $\tau$-invertible if it has a (necessarily unique) $\tau^{-1} \S$-module structure. Equivalently, if $\tau \colon \Sigma^{0, -1} X \to X$ is an equivalence.
\end{defi}

\begin{eg}
  For any $X \in \Sp$, the spectral Yoneda embedding $Y(X)$ is $\tau$-invertible.
\end{eg}

In fact, every $\tau$-invertible synthetic spectrum is of this form:
\begin{thm}
  The spectral Yoneda embedding $Y \colon \Sp \to \Syn_E$ is fully faithful with essential image given by $\tau$-invertible synthetic spectra. Further, there is a natural equivalence
  \[
    Y(X) \cong \tau^{-1} \nu X.
  \]
\end{thm}

Now consider $C\tau \otimes \nu X \cong \nu X / \tau$. If $X$ is $E$-injective, then $\nu X = y(X)$. So
\[
  [\nu A, \nu X / \tau] = \pi_0 F(A, X) = \Hom_{E_*E}(E_*A, E_* X).
\]
Given a general $X$, we can resolve it by $E$-injectives, and we find that
\begin{lemma}
  Let $A, X$ be any spectrum. Then
  \[
    [\Sigma^{a, b} \nu A, \nu X / \tau] = \operatorname{Ext}_{E_*E}^{b, a + b}(E_*A, E_*X).
  \]
\end{lemma}

In fact, it is true that
\begin{thm}[{\cite[Theorem 4.46, Proposition 4.53]{synthetic}}]
  There is a fully faithful embedding $\operatorname{Mod}_{C\tau} \to \operatorname{Comod}_{E_*E}$ that sends $\nu X$ to $E_* X$. If $E$ is Landweber exact, then this is essentially surjective.
\end{thm}
\bibliographystyle{plain}
\bibliography{syn}
\end{document}
